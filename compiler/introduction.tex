\chapter{前言}

程序语言的编译器只是另外一段程序而已。

编译器将代码从一种编程语言翻译成另外的编程语言。如果有一套新的硬件或指令集,通常编译器就是为它们而第一个被编写的程序,之后的程序再由写好的编译器进行编译。正如“道生一,一生二,二生三,三生万物”。编译器就是那个“一”,而对程序的认识理解则是“道”。因此希望读者能从本书中获得的是对程序理解方式的变化,而不仅仅是特定的知识点。

编译器是世界上最复杂的程序之一。为了降低入门的难度,书中的章节和知识采用渐进式的讲述方式。最开始,需要处理的语言只是一个简单的计算器,之后会不断的往上增加功能,直到最终扩展为一个动态语言。与大部分的编译器书籍不同,从第一章开始,我们的程序就可以将代码直接编译为X86-64汇编代码,之后每一章亦然。

本书的编译器采用一种叫“Nano Pass”的结构。从源码开始,经过多轮的处理,每次的处理结果更加的接近于汇编语言,每次的处理称为一个“pass”。每个“pass”的功能相对精简,便于掌握。

本书提供的代码,需要python3与gcc x64的支持。在64位机器上,使用较新版本的64位linux,即是完整的实验环境。核心代码采用python3进行编写,只使用python的基础语法,不使用任何外部的库,降低理解成本。后续章节需要基础的C语言知识。书中不会介绍c与python语法,无基础的同学需要提前阅读其它优秀的书籍打好基础。读者无需汇编代码的基础,涉及的知识和语法会在使用前进行讲解。

编译前的语言采用lisp的一个子集,最简单的语法,最少的关键字,简化从源码到抽象语法树的难度以及后续编写代码的工作量。此外本书代码不考虑运行效率,以简洁易懂为目标。

程序员或许与庄子心有相通之处。世间万物存于脑海,思考与编写代码时“以神遇而不以目视,官知止而神欲行”。而比庄子幸福的是,程序员可以将心中的世界通过代码表示出来,交由机器来变为现实。希望本书能成为一叶扁舟,帮助读者从此可以在代码之海中逍遥遨游。

接下来,让我们开始漫长而又快乐的旅途吧!


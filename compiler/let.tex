\chapter{变量绑定}

本章讨论变量绑定的问题,需要处理的语言记为\textbf{Let}。与大多数编程语言一致,\textbf{Let}语言采用词法作用域(lexical scoping),也就是静态作用域。此外,本章最后还为\textbf{Let}语言增加了一个最简单的运行时(runtime)环境。

\section{let语言}

下面是\textbf{Let}语言的一个样例。先将21赋值给a,然后将a赋值给b,最后返回a加b的结果42。通常为了美观和清晰,这种括号语言的缩进是在换行之后空两格。

\begin{lstlisting}[%
  language=lisp,
  caption={Let语言样例}
]
# a = 21; b = a; return a + b;
(let (a 21)
  (let (b a)
    (+ a b)))
=> 42
\end{lstlisting}


\textbf{Let}语言的语法描述如下。相比于\textbf{Adder}语言,新增了一个“let”表达式,用来将Exp的值绑定到一个变量上。幸运的是,上一小节的scan\&parse过程对\textbf{Let}语言仍然适用,因此我们可以直接从语法树开始处理。

\begin{equation}
\begin{aligned}
  \label{eq:1}
   Program \quad &::= \quad Exp \\
   Exp \quad &::= \quad (+ \quad Exp \quad Exp) \\
   Exp \quad &::= \quad (let \quad (var \quad varExp) \quad bodyExp) \\
   Exp \quad &::=  \quad Int
\end{aligned}
\end{equation}

在\textbf{Let}语言中。所有的变量都是通过let来引入的。对于letExp来说,其中的varExp继承父表达式的变量绑定关系,在bodyExp中var会覆盖父表达式中的同名变量。对于加法而言,其中的两个表达式分别继承父表达式的变量绑定规则。

\begin{lstlisting}[%
  language=lisp,
  caption={变量作用域示例}
]
# 变量a多次绑定
(let (a 1)
  (+ (let (a (+ a 1))
       a)
	a))
=>
# 对变量重命名,语义等价
(let (a.1 1)
  (+ (let (a.2 (+ a.1 1))
       a.2)
	a.1))
\end{lstlisting}

\begin{figure}[ht]
\centering
\includegraphics[scale=0.3]{figs/c2/pic1.png}
\caption{变量作用域示例}
\label{fig:fig2_1}
\end{figure}


\section{名称归一}

\section{语法树打平}

\section{选择合适的x86代码}

\section{运行时}

\begin{lstlisting}[%
  language=c++,
  caption={runtime.c}
]
#include <stdint.h>
#include <stdio.h>

void print_int(int64_t x) {
  printf("%lld\n", x);
}
\end{lstlisting}

\section{小节}